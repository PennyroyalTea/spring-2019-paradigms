% !TeX root = 190213.tex
\section{Язык программирования Python}

\begin{frame}[t]{Python?}
	\begin{itemize}
		\item Мультипарадигмальный язык общего назначения
		\item Модный, много где используется:
			\begin{itemize}
				\item Сайты (сервера)
				\item Прототипирование (анализ данных, машинное обучение)
				\item Небольшие скрипты (очень рекомендую!)
				\item Есть куча библиотек на любые случаи жизни
			\end{itemize}
		\item Где не стоит:
			\begin{itemize}
				\item Большие проекты
				\item Важна скорость или память и нет библиотеки
			\end{itemize}
		\item Есть старая версия (Python 2), есть новая (Python 3)
		\item Синтаксис несовместим, учим новую
		\item Среды разработки: Wing IDE, PyCharm (бесплатно для студентов), любой текстовый редактор и консоль
		\item Строгая динамическая типизация (в C++ "--- слабая статическая)
		\item Время для демо и практики!
	\end{itemize}
\end{frame}

\begin{frame}[t]{<<Примитивные>> типы}
	Значение копируется:
	\svgimg{int-value}
\end{frame}

\begin{frame}[t]{<<Ссылочные>> типы}
	Изменяется только ссылка:
	\svgimg{int-ref}
\end{frame}

\begin{frame}[t]{Ссылочные типы}
	Можно изменить объект по ссылке:
	\svgimg{list-ref}
\end{frame}

\begin{frame}[t,fragile]{Иммутабельные ссылочные типы}
	Объект по ссылке изменить нельзя.
	
	\verb~a += "z"~ для \textit{строк} эквивалентно \verb~a = a + "z"~

	\svgimg{str-ref}
	Всё как с <<примитивными>> типами.
\end{frame}

\begin{frame}[t]{\texttt{is} и \texttt{==}}
	\svgimg{int-ref-is}
	В случае с \texttt{int} интерпретатор может оптимизировать и не создавать
	лишние объекты для \texttt{int}.
	
	А может не оптимизировать.
\end{frame}
