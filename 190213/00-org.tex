% !TeX root = 190213.tex
\section{Организационное}
\begin{frame}[t]{Основное}
	\begin{itemize}
	\item Я выпускник бакалавриата СПб АУ
	\item Имена: <<извините, пожалуйста>>, <<Егор>>, <<Егор Фёдорович>>
	\item Можно на <<вы>>, можно на <<ты>>
	\item В любой момент можно перебить словом <<Вопрос!>> и задать \textit{любой} вопрос:
		\begin{itemize}
		\item Повторить ещё раз всю лекцию, потому что уснул
		\item Вопросы по текущей теме или смежным можно задавать в ходе рассказа на занятии
		\item Вопросы по остальным темам и предметам лучше в оффлайне
		\end{itemize}
	\item Как насчёт видеозаписи? Доступ у ограниченного круга лиц
	\item Мои контакты есть на CSCWiki
	\item Оповещения по курсу и группе "--- в рассылку и на CSCWiki
	\end{itemize}
\end{frame}

\begin{frame}[t]{Зачем курс}
	\begin{itemize}
		\item Писать много-много кода, ошибаться
		\item Перенимать от меня опыт через code review:
			\begin{enumerate}
			\item Пишете код
			\item \label{improve} Я говорю, как улучшить
			\item Вы улучшаете
			\item См. п.\ref{improve}
			\end{enumerate}
		\item Развить <<чувство прекрасного>> с точки зрения кода
		\item Рассказать про разные стили/парадигмы кода
		\item Показать, какие бывают полезные технологии
	\end{itemize}
\end{frame}

\begin{frame}[t]{Зачёт и проверка}
	\begin{itemize}
		\item Правила зачёта подробно есть на CSCWiki
		\item Посещаемость неважна, экзамена нет, только домашки
		\item Надо написать код. Оцениваются:
			\begin{enumerate}
			\item Объективно: корректность и \textit{точное} соответствие заданию "--- 50
			\item Субъективно: идиоматичность, красота кода (<<стиль>>) "--- ещё 50
			\end{enumerate}
		\item Надо сделать хоть что-нибудь в каждой домашке
		\item Надо набрать хотя бы половину баллов от максимума
		\item Обычно: одна домашка "--- одна тема (темы почти независимы)
		\item В некоторых домашках будет несколько подзаданий
		\item На домашку чуть меньше недели
			\begin{itemize}
				\item После первой недели "--- только 50\% баллов
				\item После второй недели "--- 0\% баллов, но досдавать нужно
			\end{itemize}
		\item Правила можно обсуждать со мной
	\end{itemize}
\end{frame}

\begin{frame}[t]{Списывание}
	\begin{itemize}
		\item Подробнее "--- на CSCWiki
		\item Домашки совпадают с другой группой
		\item \textbf{Курс на <<работу руками>> и чтобы вы сами что-то делали}
		\item Обсуждать можно, но не стоит сдавать чужой код или набирать код за кого-нибудь
		\item Если что-то не успеваете "--- сообщите заранее
	\end{itemize}
\end{frame}

\begin{frame}[t]{Обратная связь}
	\begin{itemize}
		\item
			Готов обсуждать и даже менять по согласованию критерии оценки, правила игры, оргмоменты.
		\item
			Любая критика и жалобы на жизнь также приветствуются.
			Особенно если есть предложения <<как лучше>>.
		\item
			Можно писать и передавать коллективные письма.
		\item
			О планируемых завалах (неделя коллоквиумов/презентация проектов/отдых) лучше предупреждать заранее.
		\item
			Кому (не)комфортно читать технический английский?
		\item
			Делитесь тайными знаниями не только с товарищами, но и со мной.
			Тогда я знаю, что я упустил на паре.
		\item Я могу начать приходить до пары, чтобы отвечать на вопросы и проверять домашки ещё быстрее
	\end{itemize}
\end{frame}

\begin{frame}[t]{Структура занятия}
	\begin{itemize}
		\item Презентация "--- на CSCWiki и в репозитории до занятия
		\item Там же до занятия лежит план рассказа
		\item После занятия могут быть небольшие обновления
		\item На паре также буду показывать код, он тоже в репозитории
		\item Если на паре придумал новый код, могу забыть выложить
		\item Можно не приходить, опаздывать, уходить и входить
		\item Лучше не шуметь
		\item Подробнее "--- на CSCWiki
		\item Вопросы "--- в любой момент
	\end{itemize}
\end{frame}
